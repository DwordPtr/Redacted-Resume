%Bryan Tidwell latex resume
\documentclass[11pt,oneside]{article}
\usepackage{geometry}
\usepackage{hyperref}
\usepackage[T1]{fontenc}

\pagestyle{empty}
\geometry{letterpaper,tmargin=1in,bmargin=1in,lmargin=.75in,rmargin=1in,headheight=0in,headsep=0in,footskip=.3in}

\setlength{\parindent}{0in}
\setlength{\parskip}{0in}
\setlength{\itemsep}{0in}
\setlength{\topsep}{0in}
\setlength{\tabcolsep}{0in}

% Name and contact information
\newcommand{\name}{Bryan Tidwell}
\newcommand{\email}{btidwell4@my.apsu.edu}
%%%%%%%%%%%%%%%%%%%%%%%%%%%%%%%%%%%%%%%%%%%%%%%%%%%%%%%%%
% New commands and environments

% This defines how the name looks
\newcommand{\bigname}[1]{
	\begin{center}\fontfamily{phv}\selectfont\Huge\scshape#1\end{center}
}

% A ressection is a main section (<H1>Section</H1>)
\newenvironment{ressection}[1]{
	\vspace{4pt}
	{\fontfamily{phv}\selectfont\Large#1}
	\begin{itemize}
	\vspace{3pt}
}{
	\end{itemize}
}

% A resitem is a simple list element in a ressection (first level)
\newcommand{\resitem}[1]{
	\vspace{-4pt}
	\item \begin{flushleft} #1 \end{flushleft}
}

% A ressubitem is a simple list element in anything but a ressection (second level)
\newcommand{\ressubitem}[1]{
	\vspace{-1pt}
	\item \begin{flushleft} #1 \end{flushleft}
}

% A resbigitem is a complex list element for stuff like jobs and education:
%  Arg 1: Name of company or university
%  Arg 2: Location
%  Arg 3: Title and/or date range
\newcommand{\resbigitem}[3]{
	\vspace{-5pt}
	\item
	\textbf{#1}---#2 \\
	\textit{#3}
}

% This is a list that comes with a resbigitem
\newenvironment{ressubsec}[3]{
	\resbigitem{#1}{#2}{#3}
	\vspace{-2pt}
	\begin{itemize}
}{
	\end{itemize}
}

% This is a simple sublist
\newenvironment{reslist}[1]{
	\resitem{\textbf{#1}}
	\vspace{-5pt}
	\begin{itemize}
}{
	\end{itemize}
}



%%%%%%%%%%%%%%%%%%%%%%%%%%%%%%%%%%%%%%%%%%%%%%%%%%%%%%%%%
% Now for the actual document:

\begin{document}

\fontfamily{ppl} \selectfont

% Name with horizontal rule
\bigname{\name}

\vspace{-8pt} \rule{\textwidth}{1pt}

\vspace{-1pt} {\small\itshape \hfill \email} \\ \\
\vspace{-1pt}{\url{https://www.linkedin.com/in/bryantidwell2}} \\
\vspace{-1pt}{\url{https://github.com/DwordPtr}}
\vspace{8 pt}

%%%%%%%%%%%%%%%%%%%%%%%%
\begin{ressection}{Experience}

	\begin{ressubsec}{Franklin American Mortgage Company}{Franklin, TN}{Developer I Mar 2016 - Present}
		\ressubitem{Contributed to MockRunner.}
		\ressubitem{Maintained business critical legacy Java apps.}
		\ressubitem{Worked on Spring Boot microservice to integrate customer data with Salesforce.}
	\end{ressubsec}


	\begin{ressubsec}{Austin Peay Computer Science Dept}{Clarksville, TN}{Lab Manager Aug 2015 - Dec 2015}
		\ressubitem{Tutored students in Math and Computer Science concepts}
		\ressubitem{Provided assistance to students setting up development enviornment config files, importing libraries, IDE projects, Unix commands, etc.}
	\end{ressubsec}

	\begin{ressubsec}{Comdata}{Brentwood, TN}{Business Objects Analyst (Intern then 1099 contractor) May 2014 - Dec 2014}
		\ressubitem{Performed QA on various BI reports and ETL jobs}
    \ressubitem{Used scripting tools to extend automated tests}
    \ressubitem{Retained as contractor on a mostly remote basis, after the internship.}
	\end{ressubsec}

\end{ressection}

%%%%%%%%%%%%%%%%%%%%%%%%
\begin{ressection}{Education}

	\begin{ressubsec}{Austin Peay State University}{Clarksville, TN}{Major in Computer Science with Minors in Mathematics and Philosophy Dec, 2015}
		\ressubitem{GPA: 3.65 \textit{Cum Laude} ( $\approx 3.8$ Computer Science GPA)}
	\end{ressubsec}
\end{ressection}

\begin{ressection}{Skills}
\begin{reslist}{Programming Languages:}
  \ressubitem{Proficient with: Java, Sql} 
  \ressubitem{Exposure: C++, Python, C\#, ML, Prolog, Javascript, TypeScript, Ruby}
\end{reslist}
\begin{reslist}{Tools}
	\ressubitem{Proficient with: git, maven, Intellij, Spring/Spring Boot, REST, JPA zsh/bash, Atlassian tooling, etc}
	\ressubitem{Exposure: Docker, Angular, Vagrant }
\end{reslist}
\end{ressection}

\begin{ressection}{Awards}
  \resitem{George Brotherton Memorial Scholarship}
\end{ressection}
\end{document}
